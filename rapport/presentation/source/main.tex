\documentclass[12pt]{beamer}
\usetheme{Madrid}
\usecolortheme{whale}
\usepackage[utf8]{inputenc}
\usepackage[T1]{fontenc}
\usepackage[french]{babel}
\usepackage{tikz}
\usetikzlibrary{mindmap,trees,shapes.geometric,arrows.meta,decorations.pathreplacing,calc,shadows,positioning,patterns,backgrounds}
\usepackage{xcolor}
\usepackage{graphicx}
\usepackage{pgfplots}
\pgfplotsset{compat=1.18}
\usepackage{tcolorbox}
\tcbuselibrary{skins}
\usepackage{multicol}
\usepackage{booktabs}
\usepackage{microtype}
\usepackage{animate}
\usepackage{listings}
\usepackage{fontawesome5}

% --- Couleurs personnalisées ---
\definecolor{donecolor}{RGB}{76,175,80}     % vert
\definecolor{todoColor}{RGB}{255,193,7}     % jaune
\definecolor{undonecolor}{RGB}{244,67,54}   % rouge
\definecolor{team1color}{RGB}{66,133,244}   % bleu
\definecolor{team2color}{RGB}{219,68,55}    % rouge
\definecolor{team3color}{RGB}{244,180,0}    % orange
\definecolor{team4color}{RGB}{15,157,88}    % vert foncé
\definecolor{umlcolor}{RGB}{0,102,204}      % bleu UML
\definecolor{codebg}{RGB}{240,245,250}      % fond code
\definecolor{accentcolor}{RGB}{103,58,183}  % violet accent
\definecolor{donecolor}{RGB}{0,150,0}    
\definecolor{graycircle}{RGB}{200,200,200} 
% Style pour les listings de code
\lstset{
    basicstyle=\ttfamily\small,
    backgroundcolor=\color{codebg},
    frame=single,
    frameround=tttt,
    rulecolor=\color{lightgray},
    tabsize=2,
    upquote=true,
    showstringspaces=false
}

% Configuration Beamer
\setbeamertemplate{footline}[frame number]
\setbeamertemplate{navigation symbols}{}
\setbeamertemplate{section in toc}[sections numbered]
\setbeamertemplate{subsection in toc}[subsections numbered]
\setbeamertemplate{caption}[numbered]

% Boîte personnalisée pour les membres
\newtcolorbox{membrebox}[2][]{
    enhanced,
    colback=white,
    colframe=#2,
    arc=3mm,
    boxrule=1.5pt,
    drop fuzzy shadow,
    fonttitle=\bfseries,
    title=#1
}
\usetheme{Madrid}

\usepackage{multicol}
\usepackage{fontawesome5}
\usepackage{appendixnumberbeamer}
\usepackage[french]{babel}
\usepackage[T1]{fontenc}
\usepackage[utf8]{inputenc}
\usepackage{graphicx}
\usepackage{tikz}
\usepackage{hyperref}
\usepackage{wrapfig} % pour l'image

\usepackage{array}


\title{{\large \texttt{Projet 2 de génie logiciel \og Jeu de Tron\fg{}}}}
\author{
    \textit{\textbf{équipe :}} \\
    \textbf{HADJ BENABDELMOULA} Lamia \\
    \textbf{KHELALFA} Selssabil \\
    \textbf{DIALLO} Hammady II \\
     \textbf{BAH} Oumar 
    \vspace{-0.6cm}
}
\date{\textit{\textbf{Présenté le :}} \today}

% En-tête vide
\setbeamertemplate{headline}{}

\begin{document}
    \begin{frame}
        \begin{center}
            \includegraphics[width=0.30\linewidth]{images/logo.png} \\
            \vspace{0.1cm}
            \texttt{ \large {Licence 3 Informatique}}
        \end{center}
        \vspace{-0.4cm}
        \titlepage
    \end{frame}


    % plan de la presentation
    \begin{frame}{Plan de la présentation}
        \centering
        \begin{minipage}{0.7\textwidth}
        \small
        \tableofcontents[sections={1-3}, hideallsubsections]
        \vspace{0.05cm}
        \tableofcontents[sections={4-7}, hideallsubsections]
        \end{minipage}

    
        \vspace{0.2cm}
        \begin{tikzpicture}
            \draw[fill=blue!60, draw=blue, thick, rounded corners] (0,0) rectangle (4.5,0.8);
            \node[white] at (2.25,0.4) {\small \textbf{Durée : 10 - 15 min}};
    
            \draw[fill=blue!10, draw=blue, thick, rounded corners] (4.6,0) rectangle (10.5,0.8);
            \node[black] at (7.55,0.4) {\small Présenté par toute l'équipe};
        \end{tikzpicture}
    \end{frame}

    %====================== CONTEXTE ======================
    \section{Contexte du projet}
    \begin{frame}{Présentation du projet « Jeu de Tron »}
        \begin{columns}[T]
            % Colonne texte
            \begin{column}{0.60\textwidth}
             \vspace{0.5cm}
                \begin{itemize}
                    \item Jeu \textbf{Tron} : variante multi-joueurs du jeu du serpent
                    \item Environnement sur grille avec collisions éliminatoires
                    \item Déplacements simultanés laissant des murs infranchissables
                    \item Objectif : être le dernier joueur survivant
                    \item Simulation de comportements \textbf{multi-agents}
                \end{itemize}
            \end{column}
    
            % Colonne image
            \begin{column}{0.35\textwidth}
                \centering
                \includegraphics[width=\linewidth]{images/jeu.png}
            \end{column}
        \end{columns}
    \end{frame}


    %====================== PROBLEMATIQUE & OBJECTIFS ======================
    \section{Problématique et objectifs du projet}
    
    \begin{frame}{Problématique et objectifs du projet}
    
        \begin{block}{Problématique}
            Comment analyser et évaluer les stratégies de décision collective dans un jeu de Tron
            multi-joueurs, en tenant compte des coalitions, de la taille des équipes et de la
            profondeur de recherche ?
        \end{block}
    
        \begin{tcolorbox}[colback=green!5!white,colframe=donecolor,title=Objectifs du projet]
            \begin{itemize}
                \item Modéliser un jeu de \textbf{Tron  multi-joueurs} sans intervention humaine.
                \item Gérer la \textbf{prise de décision collective} au sein d’équipes.
                \item Adapter des algorithmes de recherche à un contexte \textbf{multi-équipes}.
                \item Analyser l’impact des paramètres de jeu (taille de la grille, équipes, profondeur de recherche).
            \end{itemize}
        \end{tcolorbox}
    
    \end{frame}


    % ARCHITECTURE
    %-------------------------------------------------
      \begin{frame}{Structure du Cœur de Jeu}

    \centering
    \begin{tikzpicture}[node distance=1cm]
    
    % Bloc Modèle indépendant
    \node[fill=blue!10, rounded corners, text width=0.55\textwidth, align=center, draw=blue!50, thick, minimum height=1.2cm] (mod) 
    {\faCube\ \textbf{Modèle indépendant}};
    
    % Bloc Testable unitairement
    \node[fill=green!10, rounded corners, text width=0.55\textwidth, align=center, draw=green!50, thick, minimum height=1.2cm, below=0.5cm of mod] (test) 
    {\faVial\ \textbf{Testable unitairement}};
    
    % Bloc Extensible facilement
    \node[fill=orange!10, rounded corners, text width=0.55\textwidth, align=center, draw=orange!50, thick, minimum height=1.2cm, below=0.5cm of test] (ext) 
    {\faArrowsAlt\ \textbf{Extensible facilement}};
    
    \end{tikzpicture}

\end{frame}
    \section{Architecture et Modélisation}
    
    \begin{frame}{Pattern observateur – Découplage modèle/vue}
    \centering
    
    \resizebox{0.85\textwidth}{!}{%
    \begin{tikzpicture}[
        node distance=0.5cm,
        box/.style={rectangle, rounded corners, minimum width=3cm, minimum height=1cm,
        text centered, draw=black, thick, font=\small},
        arrow/.style={thick, ->, >=stealth, blue!60!black},
        observer/.style={ellipse, draw=green!60!black, thick,
        minimum width=2cm, minimum height=0.3cm, font=\small}
    ]
    
    % Modèle
    \node[box, fill=blue!10] (modele) {Modèle (ModeleJeu)};
    \node[box, fill=blue!5, below=2cm of modele] (modeleEcoutable) {\texttt{ModeleEcoutable}};
    
    % Observateurs
    \node[observer, fill=green!5, right=2cm of modele] (vue1) {Vue (GUI)};
    \node[observer, fill=green!5, above=0.5cm of vue1] (vue2) {GameBoardPanel};
    \node[observer, fill=green!5, below=0.5cm of vue1] (vue3) {Autres Vues};
    
    % Interface
    \node[box, fill=orange!5, above right=0.5cm and 2cm of vue1] (interface) {\texttt{EcouteurModele}};
    
    % Flèches
    \draw[arrow, dashed] (modele) -- node[midway, above, font=\tiny] {implémente} (modeleEcoutable);
    \draw[arrow, <->] (modeleEcoutable) -- node[midway, above, sloped, font=\tiny] {notifie()} (vue1);
    \draw[arrow, <->] (modeleEcoutable) -- (vue2);
    \draw[arrow, <->] (modeleEcoutable) -- (vue3);
    \draw[arrow] (vue1) -- node[midway, above, sloped, font=\tiny] {implémente} (interface);
    \draw[arrow] (vue2) -- (interface);
    \draw[arrow] (vue3) -- (interface);
    
    \end{tikzpicture}
    }
    
    \end{frame}

    %-------------------------------------------------
       
     \begin{frame}{Architecture générale MVC}
        \centering
        \includegraphics[width=1\linewidth, height=0.9\paperheight, keepaspectratio]{images/uml.png}
    \end{frame}


    \begin{frame}{Diagramme UML du Modèle}
        \begin{block}{}
           Le diagramme UML suivant représente la structure du modèle du le coeur du jeu. 
        \end{block}
        \vspace{0.4cm}
        \centering
        \begin{tcolorbox}[
        colback=blue!5!white,
        colframe=umlcolor, title=\faProjectDiagram\ Diagramme UML]
    
        Cliquez ici pour voir le diagramme :
        \bigskip
        
        \href{run:model.png}{\Large\textbf{\faImage\ model.png}}
        \end{tcolorbox}
        
    \end{frame}

    %-------------------------------------------------
    

 
   
    
    % Présentation par membre 
    \begin{frame}{Répartition des tâches réalisées par membre}
    \centering
    \renewcommand{\arraystretch}{1.20}
    
    \begin{tabular}{|m{2cm}|m{8.5cm}|}
        \hline
        \textbf{Membres} & \textbf{Implémentations réalisées} \\
        \hline
        
        \textbf{Oumar} &
        \begin{itemize}
            \item[\faCheckCircle] Une partie du modèle (le cœur du jeu)
            \item[\faCheckCircle] Le pattern Observer.
            \item[\faSpinner] Un \texttt{main} pour la démo.
        \end{itemize} \\
        \hline
        
        \textbf{Lamia} &
        \begin{itemize}
            \item[\faCheckCircle] Le pattern Stratégie (MinMax et Alpha-Beta)
            \item[\faCheckCircle] Les classes de tests pour ces algorithmes.
            \item[\faSpinner] La modélisation UML
        \end{itemize} \\
        \hline
        
        \textbf{Selssabil} &
        \begin{itemize}
            \item[\faCheckCircle] L’interface graphique
            \item[\faCheckCircle] Le contrôleur du jeu
            \item[\faSpinner] La vue
        \end{itemize} \\
        \hline
        
        \textbf{Hammady} &
        \begin{itemize}
            \item[\faCheckCircle] Une partie du modèle (le cœur du jeu)
            \item[\faCheckCircle] Les heuristiques et leurs évaluateurs
            \item[\faSpinner] L’analyse et l’expérimentation
        \end{itemize} \\
        \hline
    
    \end{tabular}
    \end{frame}

    
    % Niveau de prograssion (global et individuel)
    \section{État d'avancement global et individuel}
    \begin{frame}{Progression globale}
        \centering
        \begin{tikzpicture}
        \draw[gray!30, line width=20pt] (0,0) circle(2.8cm);
        \draw[donecolor, line width=20pt] (90:2.8cm) arc (90:324:2.8cm);
        
        \node at (0,0) {\LARGE \textbf{60-65\%}};
        \node at (0,-0.8) {\large  Valeur approximative};
        \end{tikzpicture}
    \end{frame}
    
    \begin{frame}{État d'avancement par composant}
        \centering
        \begin{tikzpicture}[y=-0.8cm]
        
        \foreach \y/\task/\perc/\color in {
        0/\textbf{Moteur de jeu}/85/donecolor,
        1/Algorithmes IA/60/donecolor,
        2/\textbf{Interface graphique}/60/donecolor,
        3/Console interactive/100/donecolor,
        4/\textbf{Système de tests}/20/donecolor,
        5/Optimisation/59/donecolor,
        6/\textbf{Documentation}/45/donecolor,
        7/L'analyse et l'expérimentation/45/donecolor}
        {
        \node[anchor=east] at (-1.4,\y) {\task};
        % Barre de fond (plus épaisse)
        \draw[gray!20, fill=gray!10] (0,\y-0.3) rectangle (4,\y+0.3);
        % Barre de progression
        \draw[\color, fill=\color!30] (0,\y-0.3) rectangle (4*\perc/100,\y+0.3);

        \node[anchor=west] at (4.15,\y) {\textbf{\perc\%}};
        }
        \end{tikzpicture}
    \end{frame}
    

    % Planning prévisionnel
    \section{Planning prévisionnel}
    \begin{frame}{Planning prévisionnel des semaines à venir}
    \centering
    
    \begin{tikzpicture}[
        arrow/.style={->, thick, black},
        smallcircle/.style={draw=graycircle, line width=3pt, circle, minimum size=1.5cm, inner sep=2pt},
        bigcircle/.style={draw=donecolor, line width=3pt, circle, minimum size=2.8cm, inner sep=2pt},
        textstyle/.style={font=\scriptsize, align=center}
    ]
    
    % Cercle central gris
    \node[smallcircle] (center) at (0,0) {};
    \node[textstyle] at (center) {\textbf{À venir}};
    
    % Cercles du haut (1,2,3)
    \node[bigcircle] (s1) at (-3,2.0) {};
    \node[textstyle] at (s1) {\textbf{Séance 1}\\ Robustesse du jeu, \\,l'expérimentation\\ Plan rapport};
    
    \node[bigcircle] (s2) at (0,2.5) {};
    \node[textstyle] at (s2) {\textbf{Séance 2}\\ Performance
    \\Tests unitaires\\ L'expérimentation};
    
    
    \node[bigcircle] (s3) at (3,2.0) {};
    \node[textstyle] at (s3) {\textbf{Séance 3}\\ Tests intégration\\ Multi-joueurs en GUI\\ Rapport};
    
    % Cercles du bas (4,5,6)
    \node[bigcircle] (s4) at (-3,-2.0) {};
    \node[textstyle] at (s4) {\textbf{Séance 4}\\Multithreading \\ Amélioration des algos\\et Optimisation\\Rapport};
    
    \node[bigcircle] (s5) at (0,-2.5) {};
    \node[textstyle] at (s5) {\textbf{Séance 5}\\ 
    Finalisation de \\l'interface graphique\\et Multithreading};
    
    \node[bigcircle] (s6) at (3,-2.0) {};
    \node[textstyle] at (s6) {\textbf{Séance 6}\\
    
    Finalisation des \\testes \\ Nettoyage code\\ Rapport final};
    
    % Flèches du centre vers les cercles
    \foreach \s in {s1,s2,s3,s4,s5,s6} {
        \draw[arrow] (center) -- (\s);
    }
    
    \end{tikzpicture}
    
    \end{frame}

    
    % Démonstrations et résultats
    \section{Démonstrations et résultats}
    \begin{frame}{Démonstrations et résultats}
        \begin{enumerate}
            \item \textbf{\large  Démonstration avec l'interface graphique.}
            \item \textbf{\large Démonstration avec la console}
            \item \textbf{\large L'analyse et l'expérimentation}
        \end{enumerate}
    \end{frame}
    
    
    % CONCLUSION
    \section{Conclusion et perspectives}
    \begin{frame}{Conclusion et perspectives}
        \begin{itemize}
            \item[\faCheck] \textbf{Objectifs atteints} : 
                \begin{enumerate}
                    \item Moteur de jeu fonctionnel avec que des IA
                    \item L'interface graphique de la première partie
                    \item L'analyse et l'expérimentation de la première version du jeu.
                \end{enumerate}
            \vspace{0.7cm}
            \item[\faLightbulb] \textbf{Perspectives} : 
                \begin{itemize}
                    \item Intégration de l'algorithme pour plus de 2 équipes.
                    \item Amélioration de l'interface graphique en utilisant des patterns.
                \end{itemize}
            \vspace{1cm}
            \centering
            \LARGE {\item[\faQuestionCircle] \textbf{Questions ?}}
        \end{itemize}
    \end{frame}

    
   % \begin{frame}{Difficultés rencontrées et solutions}
    % Défis et solutions avec icônes, colonnes
    %\end{frame}
    
    % ANNEXES
    %\appendix
    %\begin{frame}{Annexe : Détails techniques des heuristiques}
    % Heuristique, optimisations et métriques
    %\end{frame}
    
    %\begin{frame}{Annexe : Tests (unitaires, performance et autres)}
    % Tests unitaires, performance et intégration
  

\end{document}
                                                                                                                                                                                                                                                                                                                                                                                                                                                                                                                                                                                                                                                                                                                                                                                                                                                                                                                                                                                                                                                                                                                                                                                                                                                                                                                                                                                                                                                                                                                                                                                                                                                                                                                    